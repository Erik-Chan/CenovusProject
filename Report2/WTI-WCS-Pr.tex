\documentclass[11pt]{m2pi}
\usepackage{float}
\usepackage{hyperref}
\include{m2pi_standard_commands}
% \newcommand --------------------------------------------------------------
\newcommand{\eps}{\varepsilon}
\newcommand{\vf}{\varphi}
\newcommand{\cP}{\mathcal{P}}
\newcommand{\cH}{\mathcal{H}}
\newcommand{\cL}{\mathcal{L}}
\newcommand{\cM}{\mathcal{M}}
\newcommand{\cB}{\mathcal{B}}
\newcommand{\cD}{\mathcal{D}}
\newcommand{\cF}{\mathcal{F}}
\newcommand{\cO}{\mathcal{O}}
\newcommand{\cS}{\mathcal{S}}
\newcommand{\cN}{\mathcal{N}}
\newcommand{\cT}{\mathcal{T}}
\newcommand{\cU}{\mathcal{U}}
\newcommand{\fD}{\mathfrak{D}}
\newcommand{\fX}{\mathfrak{X}}
\newcommand{\fS}{\mathfrak{S}}
\newcommand{\bF}{\mathbb{F}}
\newcommand{\bH}{\mathbb{H}}
\newcommand{\bP}{\mathbb{P}}
\newcommand{\bR}{\mathbb{R}}
\newcommand{\bN}{\mathbb{N}}
\newcommand{\bS}{\mathbb{S}}
\newcommand{\bL}{\mathbb{L}}
\newcommand{\sF}{\mathscr{F}}
\newcommand{\sJ}{\mathscr{J}}
\newcommand{\sL}{\mathscr{L}}
\newcommand{\sS}{\mathscr{S}}
\newcommand{\sB}{\mathscr{B}}
\newcommand{\sA}{\mathscr{A}}
\newcommand{\sH}{\mathscr{H}}
\newcommand{\sP}{\mathscr{P}}
\newcommand{\la}{\langle}
\newcommand{\ra}{\rangle}
\newcommand{\intl}{\interleave}
\newcommand{\rrow}{\rightarrow}
\newcommand{\tbf}{\textbf}
\newcommand{\ves}{\varepsilon}

\begin{document}

\title{Modelling Canadian heavy crude congestion pricing}

\author{Wenning Wei}
\address{University of Calgary, Dept. of Math. and Stat.
2500 University Drive NW, Calgary AB, T2N 1N4, Canada}
\email{wenning.wei@ucalgary.ca}

\author{Erik Chan}
\address{University of Calgary, Dept. of Math. and Stat.
2500 University Drive NW, Calgary AB, T2N 1N4, Canada}
\email{erik.chan@ucalgary.ca}


%%%%%%%%%%%%%%%%%% FILL THIS IN %%%%%%%%%%%%%%%%%%%%%%%%%%%%%%%%
\author{Mahsa Azizi}
\address{}
\email{}
%%%%%%%%%%%%%%%%%% FILL THIS IN %%%%%%%%%%%%%%%%%%%%%%%%%%%%%%%%
\author{Xilai Fu}
\address{}
\email{}
%%%%%%%%%%%%%%%%%% FILL THIS IN %%%%%%%%%%%%%%%%%%%%%%%%%%%%%%%%
\author{S. Parisa Torabi}
\address{}
\email{sparisatorabi@gmail.com}
%%%%%%%%%%%%%%%%%% FILL THIS IN %%%%%%%%%%%%%%%%%%%%%%%%%%%%%%%%

\begin{abstract}
In this brief report, describe a model used to estimate the congestion surcharge for oil prices in a spatial transport network.
\end{abstract}


\maketitle

\section{Introduction and Motivation}
Spatial price integration in oil commodity markets are a common way to evaluate the degree of oil price integration between different geographic locations. Depending on the oil type, light or heavy, the price that a producer receives for a barrel of oil varies. In addition, the location of the producer and the refineries will impact the spatial price due to transportation costs \cite{Zhu2020}.


There are different benchmarks that are used globally in the oil markets, among them West Texas Intermediate (WTI) and Western Canada Select (WCS) are the topic of interest in this project.

Alberta produces approximately 4MM bbls/d (barrels per day) of a crude oil of which, approximately 3MM bbls/d is classified as heavy oil. The majority of WCS, 2/3, is consumed in United States(U.S.) Midwest (PADD II) and the rest in Gulf Coast (PADD III) (see Figure~\ref{fig:PADD Map}) is considered a marginal consumption point where the last barrel is consumed. Canadian crude oil can reach Gulf Coast through different pipeline options, the most common and cheapest method of transportation, or by rail, the most expensive. The transportation cost of the last barrel to Gulf Coast from Alberta sets the price of the Canadian heavy benchmark (WCS) in Alberta.
\begin{figure}
    \centering
    \includegraphics[width = \linewidth]{pipemap.png}
    \caption{Map of crude oil pipelines~\cite{pipelineMap}. See \href{https://www.capp.ca/explore/oil-and-natural-gas-pipelines}{{\color{blue}here}} for higher resolution.}
    \label{fig:PADD Map}
\end{figure}

The total off-take capacitance from Alberta is on average less than the supply that leads to the requirement of call on rail or shut-ins. The price response to this oversupply situation is significant and appears as a regular breakdown in the price relationship of the Gulf Coast-Alberta spread price that can be recalled as congestion. This is due to the fact that transient "sub-markets" can form in a capacitated transport system.
In this project with a collaboration with Cenovus Energy Inc., an oil and natural gas company headquartered in Alberta, we build a mixed integer linear (MIL) model to take the spatial price spread and return the congestion over a specified time horizon.

We begin by first studying a simplified version of the problem by considering only the spatial prices for Hardisty and Cushing in 2018--2020. In the next step, an Ornstein--Uhlenbeck (OU) stochastic process is fitted to the price differences between WCS and WTI to further model the price dynamics in the absence of data.

This report consists of the following sections. In section two a brief discussion on the surcharge estimation model is presented. In section three the results are discussed. In section four the reliability of the model is tested using available documented disruption. In section five we describe the network with the addition of a pseudo location. 

\section{Model}
In \cite{Zhu2020}, the authors get the conclusion that in an equilibrium integrated market, the price difference of a commodity at different locations is composed of the transportation cost and the congestion surcharge. 
\begin{lemma}[Proposition 3, \cite{Zhu2020}]
The set of equilibrium prices $\xi_s^t$ for $s\in\cS,\,t\in\cT$ over a market with fixed network structure and link costs can be expressed as 
$$\xi_s^t = \eta^t+\rho_s+\ves_s^t +\omega_s^t,\quad \forall s\in\cS,\,t\in\cT,$$
where $\ves_s^t\in[-\alpha_s, \alpha_s]$ being the neutral bound and $\omega_s^t\geq 0$ being the congestion surcharge. 
\end{lemma}
$\eta^t$ in the lemma can be the price at the a root location which is not affected by the congestion in the transportation network. We choose the price at the producer as $\eta_t$. 

Denote $\lambda_s^t$Using the surcharge estimation model (SEM) in \cite{Zhu2020}, we have the following mixed-integer and linear programming problem {\color{red} all symbols must be defined $\mathcal{S}$, $\mathcal{T}$ etc.}
\begin{align}\label{1}
\text{minimize: } &\sum_{s\in\cS} \alpha_s\\
\text{subject to: } &\lambda_s^t = \rho_s + \ves_s^t +\omega_s^t,\quad \forall \nonumber s\in\cS,t\in\cT,\\ \nonumber
&-\alpha_s\leq\ves_s^t \leq \alpha_s,\quad \forall s\in\cS,t\in\cT,\\ \nonumber
&0\leq\omega_s^t\leq \gamma_s^t M,\quad \forall s\in\cS,t\in\cT,\\ \nonumber
&\ves_s^t \geq \alpha_s - (1-\gamma_s^t) M,\quad \forall s\in\cS,t\in\cT,\\ \nonumber
&\psi^t\leq\sum_s \gamma_s^t\leq |\cS|\, \psi^t, \quad \forall t\in\cT,\\ \nonumber
&\sum_t \psi^t \leq \beta T, \quad \forall t\in\cT,\\ \nonumber
&\psi^t, \gamma_s^t \in\{0,1\}, \quad \forall s\in\cS,t\in\cT,\\ \nonumber
\end{align}

where $\lambda_{s}^{t}$ is the price spread of customer node $s$ and producer, M is a number big enough, $\beta\in[0,1]$ is a predetermined parameter representing the proportion of the time period that the surcharge term $\omega_s^t$ can take positive value.

When there is only one customer node, the above programming problem reduces to

\begin{align}
\text{minimize: } &\alpha\\\label{2} \nonumber
\text{subject to: } &\lambda^t = \rho + \ves^t +\omega^t, \quad \forall t\in\cT,\\\nonumber
&-\alpha\leq\ves^t \leq \alpha,\\\nonumber
&0\leq\omega^t\leq \psi^t M,\\\nonumber
&\ves^t \geq \alpha - (1-\psi^t) M,\\ \nonumber
&\sum_t \psi^t \leq \beta T,\\\nonumber
&\psi^t \in\{0,1\}.\nonumber
\end{align}


\section{Result}
We take Hardisty as the producer node with WCS as the price, and Cushing the customer node with WTI-5 as the price, with \$5/bbl {\color{red} is this USD? CAD? A style should be chosen at the offset and kept consistent throughout.} as the heavy/light spread. Here there is only one customer node. Using the programming problem \eqref{2}, we get the following result.

\begin{figure}
    \centering
    \includegraphics[width = \linewidth]{Interpolated prices for WTI and WCS.png}
    \caption{The price of WTI and WCS from January, 2018, to August, 2020. This data interpolates missing data by using the average of the nearest neighbours. Further, on April 20, 2020, the price of WTI hit a historical low of -37USD per barrel. This data point was treated as an outlier and removed.}
    \label{fig:my_label}
\end{figure}



With the data of WCS and WTI during the time interval Jan. 2018 to Jul. 2020, we get the following result for different value of $\beta$.

\begin{figure}
    \centering
    \includegraphics[width = \linewidth]{Estimated surcharge grouped by beta.png}
    \caption{Estimated congestion surcharges as a function of $\beta$.Notice that as $\beta$ increases, the model increases the allowable region for a surcharge to appear.}
    \label{fig:estsurcharge}
\end{figure}



\section{Documented disruption}
From January to July 2018, in western Canada, production increased steadily while export capacity remained at 2016 levels or below leading to large price discounts for Canadian crude benchmarks. As a result, WCS has averaged $US\$21.86$ lower than WTI, an increase in the differential of $71\%$ over the same period in 2017. While some discount should be expected based on quality and transportation costs to American refineries compared to other light crude streams, this normally ranges between 10USD and 15USD. Western Canadian heavy oil production increased by $9.8\%$ in 2017, and in the first half of 2018 was $8\%$ higher year-over-year. Without incremental pipeline capacity available, some production has shifted to rail, which is more expensive.%\cite{bibid} %{\color{red} What is cite\{bibid\} ??} 

In fall 2018, the WCS--WTI differential widened further than normal. This means that Canada received a much lower price for its oil than normal and many market participants were negatively affected. The primary factor behind these wide oil price differentials was a growing supply of western Canadian oil production to 4.30 million barrels per day (b/d) in September 2018. Takeaway capacity on existing pipeline systems remained constant at around 3.95 million b/d. In addition, refinery maintenance in the United States (U.S.) Midwest, the largest export market for Canadian heavy crude oil, led to a significant reduction in demand for Canadian oil.%\cite{bibid} %{\color{red} What is cite\{bibid\} ??} 
\section{Network with a pseudo location}
Model the spread as an OU process
\[dx_t  = \alpha(\mu-x_t)\,dt + \sigma dW_t\]
By the maximum log-likelihood, we get the following parameters:  $\alpha=0.0118$,$\mu=16.275$, $\sigma=2.053$.
We use a simulated process as the spread of WCS and the price at the pseudo location

%\includegraphics[height=2in, width= 6in]{simulatedpath}

Using the mixed integer programming problem described in \ref{1} with two consumer nodes Cushing and the pseudo location, we get the following result.
\section{Conclusion}

\section*{Acknowledgements}
Thank people here.

\bibliographystyle{amsplain}
\bibliography{project}

\end{document}
